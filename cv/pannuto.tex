\documentclass{article}
\usepackage[margin=1in]{geometry}

\usepackage[%
  backend=biber,
  %style=numeric,
  sorting=none,
  minnames=15,
  maxnames=25,
  defernumbers=true,
]{biblatex}

\usepackage{bold-extra}   % Provides bf+sc (only in textbf+textsc env.)
\usepackage{calc}         % Do math in tex (used for percent calculation)
\usepackage{comment}      % Provides \begin,\end{comment} for large blocks
\usepackage[protrusion=true,expansion=true,kerning,spacing]{microtype} % better type, spacing
\usepackage{tabularx}     % Complicated table creation
\usepackage[sc]{titlesec}
\usepackage{url}          % Pretty printing of hyperlinks
\usepackage[usenames,dvipsnames,svgnames]{xcolor} % Allow the use and definition of colors
\usepackage{xparse}       % For doing math (acceptance percentage in particular)
\usepackage{xstring}

% The hyperref package must loaded last. Can conflict with some packages, see:
% README ( http://ctan.mackichan.com/macros/latex/contrib/hyperref/README.pdf )
\usepackage[colorlinks=true,citecolor=violet,urlcolor=blue]{hyperref}     % Creates hyperlinks from ref/cite
\hypersetup{pdfstartview=FitH} % Sets default zoom to 100% width

% Break URLs properly (thanks to Alex Halderman)
\def\UrlBreaks{\do-\do\.\do\@\do\\\do\!\do\_\do\|\do\;\do\>\do\]\do\)\do\,\do\?\do\'\do+\do\=\do\#}
\def\UrlBigBreaks{\do\:\do\/}

% Don't typset URLs in tt font
\urlstyle{same}

\addbibresource{journals.bib}
\addbibresource{conferences.bib}
\addbibresource{workshops.bib}
\addbibresource{posterdemo.bib}

% Custom bibtex keys, from
% http://tex.stackexchange.com/questions/111846/biblatex-2-custom-fields-only-one-is-working
\DeclareSourcemap{
  \maps[datatype=bibtex,overwrite=true]{
    \map{
      \step[fieldsource=acceptance-total]
      \step[fieldset=usera,origfieldval]
    }
    \map{
      \step[fieldsource=acceptance-accepted]
      \step[fieldset=userb,origfieldval]
    }
    \map{
      \step[fieldsource=acceptance-total]
      \step[fieldset=acceptance-accepted,append=true,fieldvalue=/]
      \step[fieldset=acceptance-accepted,append=true,origfieldval]
      \step[fieldsource=acceptance-accepted]
      \step[fieldset=userc,append=true,origfieldval]
    }
    \map{
      \step[fieldsource=extra]
      \step[fieldset=userd,origfieldval]
    }
  }
}

\ExplSyntaxOn
\NewDocumentCommand{\myMathFunction}{m}
{ \fp_eval:n {round((#1)*100)} }
\ExplSyntaxOff

\DeclareFieldFormat{userc}{\myMathFunction{#1}\%}

\AtEveryBibitem{%
  \csappto{blx@bbx@\thefield{entrytype}}{% put at end of entry
    \iffieldundef{usera}{%
%    \space \textbf{No annotation!}}{%
    }{%
      \space Acceptance:%
      \space\printfield{userb}~/~\printfield{usera}%
      \space(\printfield{userc}).
    }
    \iffieldundef{userd}{%
      %
    }{%
      \textbf{\color{red}\printfield{userd}.}
    }
  }
}

\begin{document}

\nocite{*}

\begin{table}
  \centering
  \begin{tabular}{c}
    \textsc{\LARGE Pat Pannuto} \\
    \\
    \textsc{\large December 12, 2016}
  \end{tabular}
\end{table}

\begin{table*}
  \centering
  \begin{tabular*}{\textwidth}{l @{\extracolsep{\fill}} r}
    Computer Science and Engineering  & Tel: +1.248.990.4548 \\
    University of Michigan, Ann Arbor & ppannuto@umich.edu \\
    4908 BBB, 2260 Hayward            & \url{http://patpannuto.com} \\
    Ann Arbor, MI 48104               & \\
  \end{tabular*}
\end{table*}

\section*{Research Overview and Interests}

\begin{itemize}
  \item[]
    My research focuses on solving the ``last inch'' problem and solving the
    challenges that stand between the burgeoning Internet of Things and
    inevitable Internet of Everything. My interests span from low-level
    details---developing new technology to meet the energy and area demands of
    millimeter systems---to large-scale global considerations---understanding
    how our network and infrastructure must scale to support the trillions of
    impending devices.
\end{itemize}

\section*{Education}

\begin{itemize}
  \item[]
    \textbf{University of Michigan}, Ann Arbor, MI (2012--present) \\
    Ph.D. Student in Computer Science (degree expected roughly summer 2018) \\
    Advisor: Prabal Dutta

  \item[]
    \textbf{University of Michigan}, Ann Arbor, MI (2007--2012) \\
    B.S.Eng in Computer Engineering
\end{itemize}

\section*{Awards and Grants}

\begin{itemize}

  \item[] University of Michigan College of Engineering Featured Graduate Student (Fall 2014)
  \item[] Qualcomm Innovation Fellowship Honorable Mention, Team Fellowship with Brad Campbell, \$50,000 (2013--2014)
  \item[] National Defense Science \& Engineering Graduate Fellowship, \$95,000 plus tuition (2013--present)
  \item[] National Science Foundation Graduate Research Fellowship, \$90,000 plus tuition (2013, declined)
  \item[] University of Michigan Department of Computer Science First-Year Fellowship (2012--2013)
  \item[] Best Undergraduate Instructor, University of Michigan, EECS (2011--2012)

\end{itemize}

\section*{Invited Talks}

\begin{itemize}
  \item[] \textbf{Ultra-Wideband and Indoor Localization}, invited talk at
    HotWireless'16
  \item[] \textbf{The Recent Past and Distant Future of [Micro-Scale] Embedded Systems},
    Keynote at NextMote: Next Generation Platforms for the Cyber-Physical Internet,
    part of the International Conference on Embedded Wireless Systems and Networks (EWSN'16)
  \item[] \textbf{PolyPoint and the First Steps Towards Ubiquitous
      Localization}, at the Student Summit on Mobility, Systems, and
    Networking, Microsoft Research
  \item[] \textbf{Sensor Systems and the Art of Effectively Deploying Sensor Networks},
    TechChange TC111: Technology for Monitoring and Evaluation
  \item[] \textbf{Embedded System Design and the Internet of Things},
    Stanford Internet of Things Industrial Research Program
  \item[] \textbf{Sensing Technologies for Data Collection and Monitoring},
    State of the Science, Development Impact Lab (DIL) and USAID Higher Education Solutions Network (HESN)
  \item[] \textbf{MBus: Enabling the Next Generation of Sensors and Systems},
    TerraSwarm Annual Meeting
\end{itemize}

\section*{Teaching}

\begin{itemize}

  \item[] Primary Instructor, EECS 398: Computing for Computer Scientists (F'16, W'16)
    \begin{itemize}
      \item[] A new class designed and built from scratch. This class attempts
        to address the experience gap that exists across the spectrum of
        incoming Computer Science students. While driven by tools (shells,
        build systems, debuggers, version control), it explores how and why
        computer scientists interface with computers differently in their
        day-to-day activities, how to apply principles learned in courses to
        everyday activities, and ultimately how to be a more efficient user of
        computing resources.
    \end{itemize}
  \item[] Undergraduate Teaching Assistant, EECS 470: Computer Architecture (W'12)
  \item[] Undergraduate Teaching Assistant, EECS 482: Introduction to Operating Systems (W'12, F'11, W'11, F'10)
  \item[] Undergraduate Teaching Assistant, EECS 373: Design of Microprocessor Based Systems (F'11, W'11)

\end{itemize}

\section*{Professional Service}

\begin{itemize}

  \item[] 2014 ACM Workshop on Visible Light Communication Systems -- Demo Co-Chair
  \item[] Recurring reviewer for IEEE Transactions on Circuits and Systems II (TCAS-II)
    \emph{2013--present}
  \item[] Recurring reviewer for IEEE Transactions on Mobile Computing (TMC)
    \emph{2014--present}
  \item[] Recurring reviewer for USAID Development Innovation Ventures (DIV)
    \emph{2015--present}
  \item[] Computer Science Engineering Graduate Student Body President
    \emph{2013--2015}
  \item[] Computer Science Engineering Student Faculty Representative
    \emph{2015--2016}

\end{itemize}

% multibib style labels with biblatex, from
% http://tex.stackexchange.com/questions/29780/
\printbibliography[keyword=journal,heading=subbibliography,title={\Large Journal Publications},prefixnumbers={J}]
\printbibliography[keyword=conf,heading=subbibliography,title={\Large Conference Publications},prefixnumbers={C}]
\printbibliography[keyword=workshop,heading=subbibliography,title={\Large Workshop Publications},prefixnumbers={W}]
\printbibliography[keyword=posterdemo,heading=subbibliography,title={\Large Posters and Demos},prefixnumbers={PD}]


%{\bf {\em Talks and Lectures}}
%
%\begin{enumerate}
%
%\item ``Sensing Technologies for Data Collection and Monitoring''. Invited
%  Talk, State of the Science, Georgetown. March 2014
%\item ``An Introduction to Git''. Invited Lecture, Michigan Hackers,
%University of Michigan. April 2012
%
%\end{enumerate}

\end{document}
