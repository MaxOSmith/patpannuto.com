\documentclass{article}
\usepackage[margin=1in]{geometry}

\usepackage[%
  backend=biber,
  %style=numeric,
  sorting=none,
  minnames=15,
  maxnames=25,
  defernumbers=true,
]{biblatex}

\usepackage{bold-extra}   % Provides bf+sc (only in textbf+textsc env.)
\usepackage{calc}         % Do math in tex (used for percent calculation)
\usepackage{comment}      % Provides \begin,\end{comment} for large blocks
\usepackage[protrusion=true,expansion=true,kerning,spacing]{microtype} % better type, spacing
\usepackage{tabularx}     % Complicated table creation
\usepackage{textcomp}     % Missing UTF characters
\usepackage[sc]{titlesec}
\usepackage{url}          % Pretty printing of hyperlinks
\usepackage[usenames,dvipsnames,svgnames]{xcolor} % Allow the use and definition of colors
\usepackage{xparse}       % For doing math (acceptance percentage in particular)
\usepackage{xspace}
\usepackage{xstring}

% The hyperref package must loaded last. Can conflict with some packages, see:
% README ( http://ctan.mackichan.com/macros/latex/contrib/hyperref/README.pdf )
\usepackage[colorlinks=true,citecolor=violet,urlcolor=blue]{hyperref}     % Creates hyperlinks from ref/cite
\hypersetup{pdfstartview=FitH} % Sets default zoom to 100% width

% Break URLs properly (thanks to Alex Halderman)
\def\UrlBreaks{\do-\do\.\do\@\do\\\do\!\do\_\do\|\do\;\do\>\do\]\do\)\do\,\do\?\do\'\do+\do\=\do\#}
\def\UrlBigBreaks{\do\:\do\/}

% Some macros that a broadly useful:
\newcommand{\uW}{{\textmu}W\xspace}
\newcommand{\uA}{{\textmu}A\xspace}
\newcommand{\um}{{\textmu}m\xspace}
\newcommand{\us}{{\textmu}s\xspace}
\newcommand{\uF}{{\textmu}F\xspace}
\newcommand{\uJ}{{\textmu}J\xspace}
\newcommand{\iic}{I$^2$C\xspace}
\newcommand{\vdd}{V$_{\textnormal{DD}}$\xspace}

% Don't typset URLs in tt font
\urlstyle{same}

\addbibresource{journals.bib}
\addbibresource{conferences.bib}
\addbibresource{workshops.bib}
\addbibresource{posterdemo.bib}

% Custom bibtex keys, from
% http://tex.stackexchange.com/questions/111846/biblatex-2-custom-fields-only-one-is-working
\DeclareSourcemap{
  \maps[datatype=bibtex,overwrite=true]{
    \map{
      \step[fieldsource=acceptance-total]
      \step[fieldset=usera,origfieldval]
    }
    \map{
      \step[fieldsource=acceptance-accepted]
      \step[fieldset=userb,origfieldval]
    }
    \map{
      \step[fieldsource=acceptance-total]
      \step[fieldset=acceptance-accepted,append=true,fieldvalue=/]
      \step[fieldset=acceptance-accepted,append=true,origfieldval]
      \step[fieldsource=acceptance-accepted]
      \step[fieldset=userc,append=true,origfieldval]
    }
    \map{
      \step[fieldsource=extra]
      \step[fieldset=userd,origfieldval]
    }
  }
}

\ExplSyntaxOn
\NewDocumentCommand{\myMathFunction}{m}
{ \fp_eval:n {round((#1)*100)} }
\ExplSyntaxOff

\DeclareFieldFormat{userc}{\myMathFunction{#1}\%}

\AtEveryBibitem{%
  \csappto{blx@bbx@\thefield{entrytype}}{% put at end of entry
    \iffieldundef{usera}{%
%    \space \textbf{No annotation!}}{%
    }{%
      \space Acceptance:%
      \space\printfield{userb}~/~\printfield{usera}%
      \space(\printfield{userc}).
    }
    \iffieldundef{userd}{%
      %
    }{%
      \textbf{\\\color{red}\printfield{userd}.}
    }
  }
}

% https://tex.stackexchange.com/questions/297087/putting-the-title-first-in-the-bibliography
\newcommand{\nameuse}[1]{%
  \def\do##1{\settoggle{blx@use##1}{#1}}%
  \dolistcsloop{blx@datamodel@names}}

\newcommand{\nameusesave}{%
  \def\do##1{%
    \providetoggle{blx@save@use##1}%
    \iftoggle{blx@use##1}{\toggletrue{blx@save@use##1}}{\togglefalse{blx@save@use##1}}%
  }%
  \dolistcsloop{blx@datamodel@names}}

\newcommand{\nameuserestore}{%
  \def\do##1{%
    \iftoggle{blx@save@use##1}{\toggletrue{blx@use##1}}{\togglefalse{blx@use##1}}%
  }%
  \dolistcsloop{blx@datamodel@names}}

\begin{document}

\nocite{*}

\begin{table}
  \centering
  \begin{tabular}{c}
    \textsc{\LARGE Pat Pannuto} \\
    \\
    \textsc{\large October 29, 2018}
  \end{tabular}
\end{table}

\begin{table*}
  \centering
  \begin{tabular*}{\textwidth}{l @{\extracolsep{\fill}} r}
    545W Cory Hall                     & Tel: +1.248.990.4548 \\
    University of California, Berkeley & ppannuto@berkeley.edu \\
    Berkeley, CA 94720                 & \url{https://patpannuto.com} \\
  \end{tabular*}
\end{table*}




\section*{Research Interests and Overview}

\begin{itemize}
  \item[]
    From mainframes to wearables Bell's law captures the march of progress,
    noting the emergence of a new computing class roughly every decade.  My
    research explores what will be required to keep enabling these next
    generations of computing, how insights from the emerging centimeter-scale
    and nascent millimeter-scale computing class can solve problems across all
    domains, and how our expectations and interactions with technology will
    shift as we begin to realize the ubiquitous computing vision.

  \item[]
    The \textbf{MBus} project considers this from an architectural
    perspective, finding fundamental area and energy constraints in current
    interconnect technologies and demonstrating that shifting system
    management into the interconnect can simplify both system and circuit
    design.

    \textbf{Slocalization}, the first FCC-compliant ultra wideband backscatter
    platform, is motivated by deployment challenges, giving one answer for how
    we might manage the deployment of millions of miniature sensors.  It
    demonstrates decimeter-accurate sub-microwatt whole-room concurrent
    localization, creates a novel integration technique to recover signals
    from far below the noise floor, and introduces the energy versus latency
    tradeoff for systems design.

    \textbf{Harmonium} explores how to efficiently design an active ultra
    wideband tag, empowering opportunistic high-fidelity tracking, and first
    introduced the bandstitching technique that allows access to the ultra
    wideband channel using widely-available narrowband frontends.

    \textbf{SurePoint} investigates diversity in the ultra wideband channel,
    efficient protocols to capture multiple independent samples, and is the
    first to demonstrate constructive interference with 802.15.4a.

    The \textbf{Tock} operating system answers questions of management,
    bringing proper process isolation to embedded systems via new hardware and
    language features that afford safety. Tock addresses fundamental
    robustness and adaptability tensions with the introduction of grants, a
    mechanism for a statically allocated kernel to safely perform dynamic
    allocations in process memory.

    \textbf{Luxapose} shows how the smartphone camera can recover data from
    visible lights using the rolling shutter effect and realize
    centimeter-accurate position and single-degree accurate orientation from
    the projection of these lights on the camera imager.

    \textbf{Opo} crafts a novel, highly efficient ultrasonic wakeup
    frontend that enables a new broadcast ranging primitive, affording
    infrastructure-free human interaction tracking with high spatio-temporal
    fidelity.

    %aiming to understand how our interaction and utilization of technology
    %will shift as computing becomes omnipresent and its operation and
    %interaction shifts from conscious action to unconscious extension of
    %perception and ability.
    %What advancements will most change how people interact with themselves,
    %the world, and one another, and what innovations facilitate these paradigm
    %shifts?
\end{itemize}




\section*{Education}

\begin{itemize}
  \item[]
    \textbf{University of California, Berkeley}, Berkeley, CA (2017--present) \\
    Ph.D. Student in Electrical Engineering (degree expected summer 2019) \\
    Advisor: Prabal Dutta

  \item[]
    \textbf{University of Michigan}, Ann Arbor, MI (2012--2017) \\
    M.Eng. in Computer Science \\
    Advisor: Prabal Dutta

  \item[]
    \textbf{University of Michigan}, Ann Arbor, MI (2007--2012) \\
    B.S.Eng. in Computer Engineering
\end{itemize}




\section*{Awards and Honors}

\hspace{\parindent} \emph{Fellowships}

%\subsection*{Fellowships}
\begin{itemize}
  \item[] \textbf{2013} Qualcomm Innovation Fellowship (Honorable Mention), joint with Bradford Campbell, \$50,000
  \item[] \textbf{2013} National Defense Science \& Engineering Graduate Fellowship (NDSEG), \$95,000 plus tuition
  \item[] \textbf{2013} National Science Foundation Graduate Research Fellowship (NSF~GRFP), \$90,000 plus tuition
  \item[] \textbf{2012} University of Michigan Department of Computer Science First-Year Fellowship
\end{itemize}

\emph{Publication Awards}

%\subsection*{Publications}
\begin{itemize}
  \item[] \textbf{2018} Best Paper Finalist, The 17th ACM/IEEE International Conference on Information Processing in Sensor Networks
  \item[] \textbf{2017} David Wessel Best Demo Award, TerraSwarm Annual Review
  %\item[] Un \textbf{2018}iversity of Michigan College of Engineering Featured Graduate Student (Fall 2014)
  \item[] \textbf{2016} IEEE Micro Top Pick in Computer Architecture
  \item[] \textbf{2016} Outstanding Poster Award, Twelfth International Nanotechnology Conference on Communication and Cooperation
  \item[] \textbf{2015} Potential for Test of Time 2025 Award, The 2nd ACM Workshop on Hot Topics in Wireless
\end{itemize}

\emph{Teaching Honors}

%\subsection*{Teaching}
\begin{itemize}
  \item[] \textbf{2017} University of Michigan Rackham Graduate School Outstanding Graduate Student Instructor
  \item[] \textbf{2017} University of Michigan College of Engineering Richard \& Eleanor Towner Prize for Outstanding Graduate Student Instructors
  \item[] \textbf{2012} Best Undergraduate Instructor, University of Michigan, EECS
\end{itemize}



\section*{Advising and Mentoring}

\begin{itemize}

  \item[] \textbf{2018} Andreas Biri, (M.S. in progress): Adaptive protocols for interaction tracking
  \item[] \textbf{2014} Noah Nuechterlein, (undergraduate independent study): Applied computer vision

\end{itemize}




\section*{Teaching Experience}

\begin{itemize}

  \item[] Primary Instructor, EECS\,398: Computing for Computer Scientists (F16, W16)
    \begin{itemize}
      \item[] A new class designed and built from scratch. This class attempts
        to address the experience gap that exists across the spectrum of
        incoming Computer Science students. While driven by tools (shells,
        build systems, debuggers, version control), it explores how and why
        computer scientists interface with computers differently in their
        day-to-day activities, how to apply principles learned in courses to
        everyday activities, and ultimately how to be a more efficient user of
        computing resources.
      \item[] This course has been adopted as part of the permanent curriculum
        at the University of Michigan as EECS\,201: Computing Pragmatics, an
        advised co-requisite for first-year EECS majors.
      \item[] \url{https://c4cs.github.io}
      \item[] \emph{In 2017, I was awarded the Rackham Graduate School Outstanding Graduate Student Instructor and the College of Engineering Richard \& Eleanor Towner Prize for Outstanding Graduate Student Instructors for this course.}
    \end{itemize}
  \item[] Graduate Teaching Assistant, EECS\,373: Design of Microprocessor Based Systems (F15, W15)
  \item[] Undergraduate Teaching Assistant, EECS\,470: Computer Architecture (W12)
  \item[] Undergraduate Teaching Assistant, EECS\,482: Introduction to Operating Systems (W12, F11, W11, F10)
  \item[] Undergraduate Teaching Assistant, EECS\,373: Design of Microprocessor Based Systems (F11, W11)

\end{itemize}




\section*{Invited Presentations}

\begin{itemize}
  \item[] \textbf{Invited Talk:} MBus: A power-aware interconnect for ultra-low power micro-scale system design, at DARPA Near Zero Power RF and Sensor Operations (N-ZERO) Program Review (2016)
  \item[] \textbf{Invited Talk:} Ultra Wideband and Indoor Localization, at HotWireless'16
  \item[] \textbf{Keynote Address:} The Recent Past and Distant Future of [Micro-Scale] Embedded Systems, at NextMote: Next Generation Platforms for the Cyber-Physical Internet,
    part of the International Conference on Embedded Wireless Systems and Networks (EWSN'16)
  \item[] PolyPoint and the First Steps Towards Ubiquitous Localization, at the Student Summit on Mobility, Systems, and Networking, Microsoft Research
  \item[] \textbf{Guest Speaker:} Sensor Systems and the Art of Effectively Deploying Sensor Networks, TechChange TC111: Technology for Monitoring and Evaluation
  \item[] \textbf{Invited Talk:} Embedded System Design and the Internet of Things, Stanford Internet of Things Industrial Research Program
  \item[] \textbf{Invited Talk:} Sensing Technologies for Data Collection and Monitoring, State of the Science, Development Impact Lab (DIL) and USAID Higher Education Solutions Network (HESN)
  \item[] MBus: Enabling the Next Generation of Sensors and Systems, TerraSwarm Annual Meeting
\end{itemize}




\section*{Professional Service}

\begin{itemize}

  \item[] 2018 ACM Workshop on Data Acquisition to Analysis (DATA\,18) -- TPC Member
  \item[] 2014 ACM Workshop on Visible Light Communication Systems -- Demo Co-Chair
  \item[] Recurring reviewer for IEEE Transactions on Circuits and Systems II (TCAS-II)
    \emph{2013--present}
  \item[] Recurring reviewer for IEEE Transactions on Mobile Computing (TMC)
    \emph{2014--present}
  \item[] Recurring reviewer for USAID Development Innovation Ventures (DIV)
    \emph{2015--present}
  \item[] Computer Science Engineering Graduate Student Body President
    \emph{2013--2015}
  \item[] Computer Science Engineering Student Faculty Representative
    \emph{2015--2016}

\end{itemize}



\section*{References}

\begin{tabular}{l l l l}
  \textbf{Prabal Dutta} & \textbf{Anthony Rowe}      & \textbf{David Blaauw}  & \textbf{Philip Levis} \\
  UC Berkeley           & Carnegie Mellon University & University of Michigan & Stanford University \\
  prabal@berkeley.edu   & agr@ece.cmu.edu            & blaauw@umich.edu       & pal@cs.stanford.edu \\
\end{tabular}



% multibib style labels with biblatex, from
% http://tex.stackexchange.com/questions/29780/
\nameusesave
\nameuse{false}
\newrefcontext[labelprefix={J}]
\printbibliography[keyword=journal,heading=subbibliography,title={\Large Journal Publications}]
\newrefcontext[labelprefix={C}]
\printbibliography[keyword=conf,heading=subbibliography,title={\Large Conference Publications}]
\newrefcontext[labelprefix={W}]
\printbibliography[keyword=workshop,heading=subbibliography,title={\Large Workshop Publications}]
\newrefcontext[labelprefix={PD}]
\printbibliography[keyword=posterdemo,heading=subbibliography,title={\Large Posters and Demos}]
\nameuserestore


%{\bf {\em Talks and Lectures}}
%
%\begin{enumerate}
%
%\item ``Sensing Technologies for Data Collection and Monitoring''. Invited
%  Talk, State of the Science, Georgetown. March 2014
%\item ``An Introduction to Git''. Invited Lecture, Michigan Hackers,
%University of Michigan. April 2012
%
%\end{enumerate}

\end{document}
