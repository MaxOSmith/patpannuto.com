\documentclass{article}
\usepackage[margin=1in]{geometry}

\usepackage{bold-extra}   % Provides bf+sc (only in textbf+textsc env.)
\usepackage{comment}      % Provides \begin,\end{comment} for large blocks
\usepackage[protrusion=true,expansion=true,kerning,spacing]{microtype} % better type, spacing
\usepackage[labeled,resetlabels]{multibib}
\usepackage{tabularx}     % Complicated table creation
\usepackage[sc]{titlesec}
\usepackage{url}          % Pretty printing of hyperlinks
\usepackage[usenames,dvipsnames,svgnames]{xcolor} % Allow the use and definition of colors

% The hyperref package must loaded last. Can conflict with some packages, see:
% README ( http://ctan.mackichan.com/macros/latex/contrib/hyperref/README.pdf )
\usepackage[colorlinks=true,citecolor=violet,urlcolor=blue]{hyperref}     % Creates hyperlinks from ref/cite
\hypersetup{pdfstartview=FitH} % Sets default zoom to 100% width

% Break URLs properly (thanks to Alex Halderman)
\def\UrlBreaks{\do-\do\.\do\@\do\\\do\!\do\_\do\|\do\;\do\>\do\]\do\)\do\,\do\?\do\'\do+\do\=\do\#}
\def\UrlBigBreaks{\do\:\do\/}

% Don't typset URLs in tt font
\urlstyle{same}

\newcites{J,C,W,PD}%
  {%
    Journal Publications,%
    Conference Publications,%
    Workshop Publications,%
    Posters and Demos
  }

\begin{document}

\begin{table}
  \centering
  \begin{tabular}{c}
    \textsc{\LARGE Pat Pannuto} \\
    \\
    \textsc{\large August 17, 2014}
  \end{tabular}
\end{table}

\begin{table*}
  \centering
  \begin{tabular*}{\textwidth}{l @{\extracolsep{\fill}} r}
    Computer Science and Engineering  & Tel: +1.248.990.4548 \\
    University of Michigan, Ann Arbor & ppannuto@umich.edu \\
    4908 BBB, 2260 Hayward            & \url{http://patpannuto.com} \\
    Ann Arbor, MI 48104               & \\
  \end{tabular*}
\end{table*}

\section*{Research Overview and Interests}

\begin{itemize}
  \item[]
    My research focuses on solving the ``last inch'' problem and solving the
    challenges that stand between the burgeoning Internet of Things and
    inevitable Internet of Everything. My interests span from low-level
    details---developing new technology to meet the energy and area demands of
    millimeter systems---to large-scale global considerations---understanding
    how our network and infrastructure must scale to support the trillions of
    impending devices.
\end{itemize}

\section*{Education}

\begin{itemize}
  \item[]
    \textbf{University of Michigan}, Ann Arbor, MI (2012--present) \\
    Ph.D. Student in Computer Science (degree expected roughly summer 2017) \\
    Advisor: Prabal Dutta

  \item[]
    \textbf{University of Michigan}, Ann Arbor, MI (2007--2012) \\
    B.S.Eng in Computer Engineering
\end{itemize}

\section*{Awards and Grants}

\begin{itemize}

  \item[] Qualcomm Innovation Fellowship Honorable Mention, Team Fellowship with Brad Campbell, \$50,000 (2013--2014)
  \item[] National Defense Science \& Engineering Graduate Fellowship, \$95,000 plus tuition (2013--present)
  \item[] National Science Foundation Graduate Research Fellowship, \$90,000 plus tuition (2013, declined)
  \item[] University of Michigan Department of Computer Science First-Year Fellowship (2012--2103)

\end{itemize}

\section*{Teaching and Honors}

\begin{itemize}

  \item[] Best Undergraduate Instructor, University of Michigan, EECS (2011--2012)
  \item[] Undergraduate Teaching Assistant, EECS 470: Computer Architecture (W'12)
  \item[] Undergraduate Teaching Assistant, EECS 482: Introduction to Operating Systems (W'12, F'11, W'11, F'10)
  \item[] Undergraduate Teaching Assistant, EECS 373: Design of Microprocessor Based Systems (F'11, W'11)

\end{itemize}

\section*{Professional Service}

\begin{itemize}

  \item[] 2014 ACM Workshop on Visible Light Communication Systems -- Demo Co-Chair
  \item[] Reviewer for IEEE Transactions on Circuits and Systems II (TCAS-II)

\end{itemize}

\nociteJ{*}
\nociteC{*}
\nociteW{*}
\nocitePD{*}

\bibliographystyleJ{unsrt}
\bibliographystyleC{unsrt}
\bibliographystyleW{unsrt}
\bibliographystylePD{unsrt}

\bibliographyJ{journal}
\bibliographyC{conf}
\bibliographyW{workshop}
\bibliographyPD{posterdemo}

%{\bf {\em Talks and Lectures}}
%
%\begin{enumerate}
%
%\item ``Sensing Technologies for Data Collection and Monitoring''. Invited
%  Talk, State of the Science, Georgetown. March 2014
%\item ``An Introduction to Git''. Invited Lecture, Michigan Hackers,
%University of Michigan. April 2012
%
%\end{enumerate}

\end{document}
