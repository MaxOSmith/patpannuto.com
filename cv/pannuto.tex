\documentclass{article}
\usepackage[margin=1in]{geometry}

\usepackage[%
  backend=biber,
  %style=numeric,
  sorting=none,
  minnames=15,
  maxnames=25,
  defernumbers=true,
]{biblatex}

\usepackage{bold-extra}   % Provides bf+sc (only in textbf+textsc env.)
\usepackage{calc}         % Do math in tex (used for percent calculation)
\usepackage{comment}      % Provides \begin,\end{comment} for large blocks
\usepackage[protrusion=true,expansion=true,kerning,spacing]{microtype} % better type, spacing
\usepackage{tabularx}     % Complicated table creation
\usepackage{textcomp}     % Missing UTF characters
\usepackage[sc]{titlesec}
\usepackage{url}          % Pretty printing of hyperlinks
\usepackage[usenames,dvipsnames,svgnames]{xcolor} % Allow the use and definition of colors
\usepackage{xparse}       % For doing math (acceptance percentage in particular)
\usepackage{xspace}
\usepackage{xstring}

% The hyperref package must loaded last. Can conflict with some packages, see:
% README ( http://ctan.mackichan.com/macros/latex/contrib/hyperref/README.pdf )
\usepackage[colorlinks=true,citecolor=violet,urlcolor=blue]{hyperref}     % Creates hyperlinks from ref/cite
\hypersetup{pdfstartview=FitH} % Sets default zoom to 100% width

% Break URLs properly (thanks to Alex Halderman)
\def\UrlBreaks{\do-\do\.\do\@\do\\\do\!\do\_\do\|\do\;\do\>\do\]\do\)\do\,\do\?\do\'\do+\do\=\do\#}
\def\UrlBigBreaks{\do\:\do\/}

% Some macros that a broadly useful:
\newcommand{\uW}{{\textmu}W\xspace}
\newcommand{\uA}{{\textmu}A\xspace}
\newcommand{\um}{{\textmu}m\xspace}
\newcommand{\us}{{\textmu}s\xspace}
\newcommand{\uF}{{\textmu}F\xspace}
\newcommand{\uJ}{{\textmu}J\xspace}
\newcommand{\iic}{I$^2$C\xspace}
\newcommand{\vdd}{V$_{\textnormal{DD}}$\xspace}

% Don't typset URLs in tt font
\urlstyle{same}

\addbibresource{journals.bib}
\addbibresource{conferences.bib}
\addbibresource{workshops.bib}
\addbibresource{posterdemo.bib}

% Custom bibtex keys, from
% http://tex.stackexchange.com/questions/111846/biblatex-2-custom-fields-only-one-is-working
\DeclareSourcemap{
  \maps[datatype=bibtex,overwrite=true]{
    \map{
      \step[fieldsource=acceptance-total]
      \step[fieldset=usera,origfieldval]
    }
    \map{
      \step[fieldsource=acceptance-accepted]
      \step[fieldset=userb,origfieldval]
    }
    \map{
      \step[fieldsource=acceptance-total]
      \step[fieldset=acceptance-accepted,append=true,fieldvalue=/]
      \step[fieldset=acceptance-accepted,append=true,origfieldval]
      \step[fieldsource=acceptance-accepted]
      \step[fieldset=userc,append=true,origfieldval]
    }
    \map{
      \step[fieldsource=extra]
      \step[fieldset=userd,origfieldval]
    }
  }
}

\ExplSyntaxOn
\NewDocumentCommand{\myMathFunction}{m}
{ \fp_eval:n {round((#1)*100)} }
\ExplSyntaxOff

\DeclareFieldFormat{userc}{\myMathFunction{#1}\%}

\AtEveryBibitem{%
  \csappto{blx@bbx@\thefield{entrytype}}{% put at end of entry
    \iffieldundef{usera}{%
%    \space \textbf{No annotation!}}{%
    }{%
      \space Acceptance:%
      \space\printfield{userb}~/~\printfield{usera}%
      \space(\printfield{userc}).
    }
    \iffieldundef{userd}{%
      %
    }{%
      \textbf{\color{red}\printfield{userd}.}
    }
  }
}

\begin{document}

\nocite{*}

\begin{table}
  \centering
  \begin{tabular}{c}
    \textsc{\LARGE Pat Pannuto} \\
    \\
    \textsc{\large October 4, 2018}
  \end{tabular}
\end{table}

\begin{table*}
  \centering
  \begin{tabular*}{\textwidth}{l @{\extracolsep{\fill}} r}
    545W Cory Hall                     & Tel: +1.248.990.4548 \\
    University of California, Berkeley & ppannuto@berkeley.edu \\
    Berkeley, CA 94720                 & \url{https://patpannuto.com} \\
  \end{tabular*}
\end{table*}

\section*{Research Overview and Interests}

\begin{itemize}
  \item[]
    My research vision pushes towards the realization of ubiquitous and
    pervasive computing,
    aiming to understand how our interaction and utilization of technology
    will shift as computing becomes omnipresent and its operation and
    interaction shifts from conscious action to unconscious extension of
    perception and ability.
    What advancements will most change how people interact with themselves,
    the world, and one another, and what innovations facilitate these paradigm
    shifts?
    My interests span from low-level details---developing new technology to
    meet the energy and area demands of next-generation millimeter
    systems---to large-scale global considerations---understanding how our
    network and infrastructure must scale and adapt to support the trillions
    of impending devices.
\end{itemize}

\section*{Education}

\begin{itemize}
  \item[]
    \textbf{University of California, Berkeley}, Berkeley, CA (2017--present) \\
    Ph.D. Student in Electrical Engineering (degree expected summer 2019) \\
    Advisor: Prabal Dutta

  \item[]
    \textbf{University of Michigan}, Ann Arbor, MI (2012--2017) \\
    M.Eng. in Computer Science \\
    Advisor: Prabal Dutta

  \item[]
    \textbf{University of Michigan}, Ann Arbor, MI (2007--2012) \\
    B.S.Eng in Computer Engineering
\end{itemize}

\section*{Awards and Honors}

\begin{itemize}

  \item[] University of Michigan Rackham Graduate School Outstanding Graduate Student Instructor (2017)
  \item[] University of Michigan College of Engineering Richard \& Eleanor Towner Prize for Outstanding Graduate Student Instructors (2017)
  \item[] University of Michigan College of Engineering Featured Graduate Student (Fall 2014)
  \item[] Qualcomm Innovation Fellowship Honorable Mention, Team Fellowship with Brad Campbell, \$50,000 (2013--2014)
  \item[] National Defense Science \& Engineering Graduate Fellowship, \$95,000 plus tuition (2013--2016)
  \item[] National Science Foundation Graduate Research Fellowship, \$90,000 plus tuition (2013, declined)
  \item[] University of Michigan Department of Computer Science First-Year Fellowship (2012--2013)
  \item[] Best Undergraduate Instructor, University of Michigan, EECS (2011--2012)

\end{itemize}

\section*{Invited Presentations}

\begin{itemize}
  \item[] \textbf{Invited Talk:} MBus: A power-aware interconnect for ultra-low power micro-scale system design, at DARPA Near Zero Power RF and Sensor Operations (N-ZERO) Program Review (2016)
  \item[] \textbf{Invited Talk:} Ultra-Wideband and Indoor Localization, at HotWireless'16
  \item[] \textbf{Keynote Address:} The Recent Past and Distant Future of [Micro-Scale] Embedded Systems, at NextMote: Next Generation Platforms for the Cyber-Physical Internet,
    part of the International Conference on Embedded Wireless Systems and Networks (EWSN'16)
  \item[] PolyPoint and the First Steps Towards Ubiquitous Localization, at the Student Summit on Mobility, Systems, and Networking, Microsoft Research
  \item[] \textbf{Guest Speaker:} Sensor Systems and the Art of Effectively Deploying Sensor Networks, TechChange TC111: Technology for Monitoring and Evaluation
  \item[] \textbf{Invited Talk:} Embedded System Design and the Internet of Things, Stanford Internet of Things Industrial Research Program
  \item[] \textbf{Invited Talk:} Sensing Technologies for Data Collection and Monitoring, State of the Science, Development Impact Lab (DIL) and USAID Higher Education Solutions Network (HESN)
  \item[] MBus: Enabling the Next Generation of Sensors and Systems, TerraSwarm Annual Meeting
\end{itemize}

\section*{Teaching}

\begin{itemize}

  \item[] Primary Instructor, EECS 398: Computing for Computer Scientists (F'16, W'16)
    \begin{itemize}
      \item[] A new class designed and built from scratch. This class attempts
        to address the experience gap that exists across the spectrum of
        incoming Computer Science students. While driven by tools (shells,
        build systems, debuggers, version control), it explores how and why
        computer scientists interface with computers differently in their
        day-to-day activities, how to apply principles learned in courses to
        everyday activities, and ultimately how to be a more efficient user of
        computing resources.
      \item[] \url{https://c4cs.github.io}
      \item[] \emph{In 2017, I was awarded the Rackham Graduate School Outstanding Graduate Student Instructor and the College of Engineering Richard \& Eleanor Towner Prize for Outstanding Graduate Student Instructors for this course.}
    \end{itemize}
  \item[] Graduate Teaching Assistant, EECS 373: Design of Microprocessor Based Systems (F'15, W'15)
  \item[] Undergraduate Teaching Assistant, EECS 470: Computer Architecture (W'12)
  \item[] Undergraduate Teaching Assistant, EECS 482: Introduction to Operating Systems (W'12, F'11, W'11, F'10)
  \item[] Undergraduate Teaching Assistant, EECS 373: Design of Microprocessor Based Systems (F'11, W'11)

\end{itemize}

\section*{Professional Service}

\begin{itemize}

  \item[] 2014 ACM Workshop on Visible Light Communication Systems -- Demo Co-Chair
  \item[] Recurring reviewer for IEEE Transactions on Circuits and Systems II (TCAS-II)
    \emph{2013--present}
  \item[] Recurring reviewer for IEEE Transactions on Mobile Computing (TMC)
    \emph{2014--present}
  \item[] Recurring reviewer for USAID Development Innovation Ventures (DIV)
    \emph{2015--present}
  \item[] Computer Science Engineering Graduate Student Body President
    \emph{2013--2015}
  \item[] Computer Science Engineering Student Faculty Representative
    \emph{2015--2016}

\end{itemize}

% multibib style labels with biblatex, from
% http://tex.stackexchange.com/questions/29780/
\printbibliography[keyword=journal,heading=subbibliography,title={\Large Journal Publications},prefixnumbers={J}]
\printbibliography[keyword=conf,heading=subbibliography,title={\Large Conference Publications},prefixnumbers={C}]
\printbibliography[keyword=workshop,heading=subbibliography,title={\Large Workshop Publications},prefixnumbers={W}]
\printbibliography[keyword=posterdemo,heading=subbibliography,title={\Large Posters and Demos},prefixnumbers={PD}]


%{\bf {\em Talks and Lectures}}
%
%\begin{enumerate}
%
%\item ``Sensing Technologies for Data Collection and Monitoring''. Invited
%  Talk, State of the Science, Georgetown. March 2014
%\item ``An Introduction to Git''. Invited Lecture, Michigan Hackers,
%University of Michigan. April 2012
%
%\end{enumerate}

\end{document}
